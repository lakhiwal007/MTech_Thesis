\chapter*{ABSTRACT}
Combustion of hydrocarbon fuels has been a major source of energy for many industries, including transportation, power generation, and heating. While hydrocarbon fuels are abundant and widely available, their combustion is associated with the emission of pollutants such as CO, NOx, UHC, and soot, which can have detrimental effects on both human health and the environment. Peripheral vortex reverse flow (PVRF) and stagnation point reverse flow (SPRF) combustors have been demonstrated to result in low pollutant emissions and stable combustion. In this study, the performance of a PVRF and an SPRF combustor is experimentally and numerically investigated. Compressed natural gas (CNG) is used as the fuel with combustor heat load of 6.25 kW. The combustor has cuboidal shape with dimensions of 80 mm × 80 mm × 40 mm with air/fuel injection ports as well as the exhaust port located at the top side to facilitate an overall reverse flow geometry. The air/fuel injection ports are located at the center of the top side. In the non-premixed mode, fuel is injected coaxially w.r.t. the air jet, whereas in the premixed mode the fuel is mixed with the air at a far upstream location to allow complete mixing before introducing the mixture into the combustor. Both air and fuel injection velocities are relatively high (30-100 m/s) to generate high turbulence and mixing. The PVRF combustor has a single exhaust on one of the sides of the air/fuel injection port that facilitates a strong peripheral vortex on the other side of the combustor. The SPRF combustor has two exhausts located on either sides of the air/fuel injection ports having gas recirculation from both the sides of the air injection. Experiments were conducted to measure NOx, CO emissions, operational limits and CH* chemiluminescence imaging is done to locate the reaction zone. Simulations were performed using RANS models and global chemical kinetic mechanisms to understand the flow field and gas recirculation pattern inside the combustor. Very low ($ < $ 5 ppm NOx at equivalence ratio of 0.6) emission levels were obtained for both SPRF and PVRF combustors in both non-premixed and premixed modes of operation. The trends for NOx and CO w.r.t. the equivalence ratio were also same for both PVRF and SPRF combustors. It was observed that the PVRF combustor resulted in slightly lower NOx emissions in the non-premixed mode as compared to the premixed mode. Numerical simulations suggest higher recirculation ratios for the PVRF combustor and compared to the SPRF combustor. The reaction zone location also varies considerably for premixed and non-premixed modes for both combustors.


\vspace*{0.7cm}
%\vspace*{1cm}
\noindent
{\large{\textbf{\emph{Keywords:}}}}~
Peripheral vortex reverse flow, Stagnation point reverse flow, Non-premixed combustion, Premixed combustion, Emissions







