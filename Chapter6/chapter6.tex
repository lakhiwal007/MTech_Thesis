\chapter[Conclusions and Future scope]{Conclusions and Future Scope}{Conclusions and Future scope}\label{CH:CFS}

\section{Conclusions}
In this work, two low emission reverse flow configurations are investigated through numerical and experimental analyses. In the numerical results, flow field characteristics, temperature field characteristics, and recirculation ratio within the combustors were investigated at a global equivalence ratio of 0.8 for non-reacting and reacting cases. The flow field and recirculation ratio results show better mixing in the PVRF combustor compared to the SPRF combustor. Temperature distribution is more uniform in SPRF combustor than PVRF combustor.  

Experimental results, including global flame images and CH* chemiluminescence, revealed that in the non-premixed mode, the reaction initiated near the bottom of the combustor, while in the premixed mode, it occurred in the center. The non-premixed mode exhibited lower spatially and temporally averaged CH* intensity but higher fluctuations. Decreasing the equivalence ratio resulted in reduced combustion chamber intensity for both combustors. NOx emissions remained below 15 ppm, with the lowest emissions recorded as 1.3 ppm in the SPRF combustor for premixed mode at an equivalence ratio of 0.5, and 1.8 ppm in the PVRF combustor for non-premixed mode at an equivalence ratio of 0.6. The lowest CO emissions of 13 ppm and 21 ppm were measured for the SPRF and PVRF combustors, respectively, in the premixed mode at an equivalence ratio of 0.55. The obtained correlation between CH* intensity and NOx emissions suggests the potential for utilizing CH* measurements as an indicator of NOx emissions, especially in premixed. 
mode

\section{Future scope}
The current study was conducted on a small-scale combustor in the laboratory, operating at a maximum heat load of 6.25 kW. However, scaling practical combustors can achieved by incorporating multiple fuel injection points. However, the presented approach still allows for the assessment of crucial practical factors such as pollutant emissions, pressure drop, combustion stability, and thermal intensity. By obtaining information on these parameters, we can gain a deeper understanding of the feasibility and performance of reverse flow combustors in real-world applications.

In future, experiments can be performed using Particle Image Velocimetry (PIV) and Planar Laser-Induced Fluorescence (PLIF). This will allows for a more comprehensive investigation of the flow field, which can then be used to validate the accuracy of the numerical results. Additionally, PLIF, can be employed to acquire detailed information about the intricate structure of the reaction zone, as well as the distribution of species and temperature. These techniques are particularly useful for studying turbulent combustion in both premixed and non-premixed modes. By utilizing PLIF, it becomes possible to identify regions of NO formation and explore innovative methods to minimize pollutant emissions.

Aircraft engines operate at even higher pressures, approximately 40 atm during takeoff and 10 atm during cruise. As a result, testing and exploring various pressure ranges can be beneficial for diverse applications in the field of gas turbine technology.